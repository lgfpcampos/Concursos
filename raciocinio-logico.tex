\chapter{Raciocínio Lógico}

\quest{Oficial de Justiça 2023 - VUNESP}{ Sabendo-se que é falsidade a afirmação “Se Nora trabalhou, então ela precisa descansar”, assinale a alternativa que apresenta uma afirmação verdadeira.}{
\item Nora trabalhou e ela não precisa descansar.
\item Nora não trabalhou e ela não precisa descansar.
\item Nora trabalhou e ela precisa descansar.
\item Nora não trabalhou ou ela precisa descansar.
\item Nora não trabalhou e ela precisa descansar.}{https://youtu.be/DQVfJCeKXF8}

\quest{Oficial de Justiça 2023 - VUNESP}
{Considere verdadeiras as seguintes premissas:
\begin{enumerate}[I.]
\item Se Carla não é casada ou Pedro não é divorciado, então Cláudio é filho único.
\item Se Sônia é mãe, então Carla não é casada.
\item Se Pedro não é divorciado, então Sergio não é administrador e Gerson é noivo.
\item Cláudio não é filho único.
\end{enumerate}
Uma conclusão que decorre das premissas apresentadas
e forma, juntamente com as premissas, um argumento
válido é
}{\item Gerson é noivo.
\item Sergio não é administrador.
\item Sônia não é mãe.
\item Sônia é mãe.
\item Sergio é administrador.}
{https://youtu.be/XXmDn4BTLn8}

\quest{Oficial de Justiça 2023 - VUNESP}
{Considere a seguinte afirmação: “Ou durmo ou trabalho”.
Uma negação lógica para a afirmação apresentada é}{
\item Ou não durmo ou não trabalho.
\item Trabalho ou durmo.
\item Se não durmo, então não trabalho.
\item Não trabalho e não durmo.
\item Durmo se, e somente se, trabalho.}
{https://youtu.be/w22pO_WZ_dk}

\quest{Oficial de Justiça 2023 - VUNESP}
{Considere verdadeira a afirmação “Se Marcelo é professor universitário, então Raquel é advogada” e falsa a afirmação “Marcelo é professor universitário e Raquel é advogada”. Nessas condições, é necessariamente verdade que}{
\item Marcelo não é professor universitário.
\item Raquel não é advogada.
\item Marcelo é professor universitário.
\item Marcelo é professor universitário ou Raquel não é advogada.
\item Raquel é advogada.}
{https://youtu.be/kQjrUqqL9eM}

\quest{Técnico Administrativo - IBFC 2023}{ Se todo vegetal é nutritivo e alguns alimentos
são nutritivos, então é correto afirmar que:}{
\item Todo alimento é vegetal
\item Não pode haver alimento que é vegetal
\item Não pode haver alimento que não é vegetal
\item Pode haver vegetal que não é alimento}
{https://youtu.be/ka6Kcb0-Vf8}

\quest{Técnico Administrativo - IBFC 2023}{Se o valor lógico de uma proposição p é verdade e o valor lógico de uma proposição q é falso, então é correto afirmar que:}{
\item A conjunção entre p e q tem valor lógico verdade
\item O bicondicional entre p e q tem valor lógico falso
\item A disjunção entre p e q tem valor lógico falso
\item A disjunção exclusiva entre p e q tem valor lógico falso}
{https://youtu.be/vkQIpeSXZCY}

\quest{Técnico Interno BBTS 2023 - FGV}{Considere a afirmação a seguir.
“Eu fiz dieta e não emagreci.”\\
A negação lógica dessa afirmação é:}{
\item Eu não fiz dieta e não emagreci.
\item Eu não fiz dieta ou emagreci.
\item Eu não fiz dieta e emagreci.
\item Eu não fiz dieta ou não emagreci.
\item Eu fiz dieta e emagreci.}
{https://youtu.be/16Po-EIt1ec}