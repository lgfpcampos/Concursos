\chapter{Função Afim}
\quest{Banco do Brasil 2023- CESGRANRIO}{Um fabricante sabe que o custo de produção de 1.000
pares de chinelos é de R\$ 8.800,00 e que o custo para a produção de 400 pares é de R\$ 4.900,00. Considere que o custo de produção C(x) de x pares de chinelos é dado pela função definida por C(x) = ax + b, em que b indica o custo fixo. Sendo assim, o custo de produção de 2.000 pares de chinelos, em reais, é igual a}
{
\item 24.500,00
\item 17.600,00
\item 15.300,00
\item 13.600,00
\item 12.400,00
}{https://youtu.be/vSWK8_rwDGk}

\quest{Transpetro 2023 - CESGRANRIO}{Em uma fábrica, há um tanque cuja capacidade máxima é de 180 m³. Estando o tanque vazio, três torneiras de mesma vazão gastam oito horas para enchê-lo completamente. Um outro tanque, com capacidade máxima de x metros cúbicos, está sendo construído e, quando vazio, cinco torneiras (com a mesma vazão das anteriores) deverão enchê-lo completamente em apenas y horas. Nessas condições, o valor de y em função de x é definido por}
{
\item y = 2x/81
\item y = 2x/54
\item y = 2x/45
\item y = 2x/27
\item y = 2x/75}
{https://youtu.be/gVBGXh3w-5k}