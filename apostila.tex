\documentclass[12pt,a4paper]{report}
\usepackage[utf8]{inputenc}
\usepackage[portuguese]{babel}
\usepackage[T1]{fontenc}
\usepackage{amsmath}
\usepackage{amsfonts}
\usepackage{amssymb}
\usepackage{graphicx}
\usepackage{fourier}
\usepackage[left=2cm,right=2cm,top=2cm,bottom=2cm]{geometry}
\author{Luiz Guilherme - Professor de Matemática}
\title{Matemática para Concursos Públicos}
\date{Última atualização:  \today \\ \qrcode{https://docs.google.com/viewer?url=https://raw.githubusercontent.com/lgfpcampos/Concursos/main/apostila.pdf}} 
\usepackage{tikz}
\usepackage{tcolorbox}
\usepackage{hyperref}
\usepackage{qrcode}
\usepackage{xcolor}
\usepackage{fontawesome}
\usepackage{enumerate}
\usepackage[Glenn]{fncychap}
\usepackage{multicol}



\begin{document}

\newcommand{\quest}[4]{ \item 
	\textbf{#1} {\color{red}\faYoutubePlay}\\ % Banca e Concurso
	 {#2}\\ %enunciado
		\begin{minipage}{12cm}
		\begin{enumerate}
			{#3} %alternativas
		\end{enumerate}
	\end{minipage}
	\qrcode{#4&list=UULFeA_dOOqAA4DB92B_BWDU7Q}} %youtube


\maketitle
\tableofcontents	

\begin{enumerate}
\chapter{Conjuntos}
\quest{Oficial de Justiça 2023 - VUNESP}{Sobre um grupo de atletas sabe-se que 15 praticam
natação, atletismo e ciclismo, 20 praticam somente n­atação e atletismo, 27 praticam somente natação e
c­iclismo, e 25 praticam somente atletismo e ciclismo. Se 70 atletas desse grupo praticam natação, 61 praticam atletismo, e 75 praticam ciclismo, então é verdade que, das alternativas a seguir, a que contém a porcentagem que mais se aproxima da relação entre o número de atletas que praticam um único esporte o número total de atletas desse grupo é}
{\item 12\%
\item 18\%
\item 20\%
\item 16\%
\item 14\%
}{https://youtu.be/SYUNxu4fcF8}
\section{Máximo Divisor Comum e Mínimo Múltiplo Comum }

\quest{Oficial de Justiça 2023 - VUNESP}{Uma empresa executa serviços aos seus clientes somente de segunda-feira a sexta-feira, independentemente de haver feriado ou não. Para seu cliente X&W, ela executa serviços a cada 12 dias, excluindo-se sábados e domingos, enquanto que para seu cliente W&Z, ela executa serviços a cada 33 dias, também excluindo-se sábados e domingos. No dia 15 de agosto de 2023, uma terça-feira, essa empresa executou serviços para ambos os clientes.
Isso significa que a vez imediatamente posterior em que ela executou os serviços para ambos os clientes, em um mesmo dia, foi uma}
{\item sexta-feira.
\item segunda-feira.
\item quinta-feira.
\item quarta-feira.
\item terça-feira.}
{https://youtu.be/o07aeZ4jtj4}



\section{Média Aritmética Simples e Ponderada}


\quest{Oficial de Justiça 2023 - VUNESP}{O gráfico apresenta o número de acertos na prova de Língua Portuguesa e de Matemática, aplicada a dois candidatos, A e B, em um concurso interno para promoção de cargo:\\
	\includegraphics[scale=.5]{fig001.png}\\
Sabendo-se que a prova de Língua Portuguesa tinha 	peso 2 e a de Matemática tinha peso 3 para o cargo em 	concurso, que cada uma das provas tinha 50 questões, 	e que a nota de cada prova é igual ao número de acertos correspondente, é correto afirmar que o número de questões de Matemática que o candidato B deveria ter acertado a mais, para que a média aritmética ponderada das notas das suas provas fosse igual à média aritmética ponderada das notas das provas do candidato A, é igual a}
{\item 9.
\item 20.
\item 10.
\item 29.
\item 27.}
{https://youtu.be/BvsQZctqarQ}

\chapter{Sequências}
\quest{Oficial de Justiça 2023 - VUNESP}{ Na sequência numérica 1, 4, 7, 8, 11,14, 19, 22, 25, 26, 29, 32, 37, ..., o 1º elemento é o número 1. Mantida a regularidade, o 11 111º elemento é o número}
{ 
\item 33 332.
\item 31 111.
\item 33 115.
\item 33 329.
\item 32 228.
}{https://youtu.be/whv3ZbhWa9U}
\chapter{Razão e Proporção}
\quest{Oficial de Justiça 2023 - VUNESP}{No ano de 2022, 3 em cada 8 edifícios comercializados em determinada região foram adquiridos pelo empreendimento A&B, que investiu R\$ 1,35 bilhão na compra desses edifícios, ao preço médio de R\$ 15 milhões cada edifício. Dos edifícios não adquiridos pelo empreendimento A&B e que foram comercializados naquela r­egião, o empreendimento R&T adquiriu metade, ao custo total R\$ 1,23 bilhão, o que fez com que o preço médio, por edifício adquirido pela R&T, fosse de}
{
\item R\$ 16,3 milhões.
\item R\$ 16,1 milhões.
\item R\$ 16,5 milhões.
\item R\$ 16,4 milhões.
\item R\$ 16,2 milhões.}
{https://youtu.be/41SHCi5jX54} 

\chapter{Regra de Três Simples e Composta}

\section{Regra de Três Composta}

\quest{Oficial de Justiça 2023 - VUNESP}
{Considere as informações apresentadas na tabela a seguir, relacionadas à produção de certa peça que é
realizada apenas por máquinas iguais, trabalhando ao mesmo tempo, com a mesma capacidade de produção.\\
\includegraphics[scale=.5]{fig002.png}\\
Sabendo-se que as informações apresentadas são proporcionais, que em 30/08/2023 o número de máquinas
em funcionamento era um quinto maior que o número de máquinas trabalhando no dia seguinte, e que o número
de peças produzidas em 31/08/2023 foi quatro terços do número de peças produzidas no dia anterior, é correto afirmar que a carga horária trabalhada no dia 31/08/2023 foi de}
{\item 8 horas.
\item 7 horas.
\item 9 horas.
\item 8 horas e 30 minutos.
\item 7 horas e 30 minutos.}
{https://youtu.be/mVp0Bf8s8J4}

\quest{Banco do Brasil 2023 - CESGRANRIO}{G máquinas idênticas imprimem G panfletos idênticos, em G dias, trabalhando G horas por dia. H máquinas idênticas às primeiras imprimem H panfletos idênticos aos primeiros, em T dias, trabalhando H horas por dia. Portanto, T é igual a
}{\item $\dfrac{H^2}{G}$
\item $\dfrac{G^3}{H}$
\item $\dfrac{H^3}{G^2}$
\item $\dfrac{G^2}{H}$
\item $\dfrac{G^2}{H^3}$ }{https://youtu.be/3GMewuLWYz0}

\quest{Técnico Interno BBTS 2023 - FGV}{Em uma fábrica de camisetas, as costureiras têm a mesma
eficiência.
Para cumprir certa encomenda 3 costureiras fizeram 100
camisetas em 4 dias. Para realizar o trabalho de nova
encomenda, 5 costureiras trabalharam 6 dias inteiros.
Assinale a opção que indica a quantidade de camisetas fabricadas
para essa segunda encomenda.}{
\item 200.
\item 215.
\item 225.
\item 240.
\item 250.}
 {https://youtu.be/bjrOtZc6GnQ}

\chapter{Raciocínio Lógico}

\quest{Oficial de Justiça 2023 - VUNESP}{ Sabendo-se que é falsidade a afirmação “Se Nora trabalhou, então ela precisa descansar”, assinale a alternativa que apresenta uma afirmação verdadeira.}{
\item Nora trabalhou e ela não precisa descansar.
\item Nora não trabalhou e ela não precisa descansar.
\item Nora trabalhou e ela precisa descansar.
\item Nora não trabalhou ou ela precisa descansar.
\item Nora não trabalhou e ela precisa descansar.}{https://youtu.be/DQVfJCeKXF8}

\quest{Oficial de Justiça 2023 - VUNESP}
{Considere verdadeiras as seguintes premissas:
\begin{enumerate}[I.]
\item Se Carla não é casada ou Pedro não é divorciado, então Cláudio é filho único.
\item Se Sônia é mãe, então Carla não é casada.
\item Se Pedro não é divorciado, então Sergio não é administrador e Gerson é noivo.
\item Cláudio não é filho único.
\end{enumerate}
Uma conclusão que decorre das premissas apresentadas
e forma, juntamente com as premissas, um argumento
válido é
}{\item Gerson é noivo.
\item Sergio não é administrador.
\item Sônia não é mãe.
\item Sônia é mãe.
\item Sergio é administrador.}
{https://youtu.be/XXmDn4BTLn8}

\quest{Oficial de Justiça 2023 - VUNESP}
{Considere a seguinte afirmação: “Ou durmo ou trabalho”.
Uma negação lógica para a afirmação apresentada é}{
\item Ou não durmo ou não trabalho.
\item Trabalho ou durmo.
\item Se não durmo, então não trabalho.
\item Não trabalho e não durmo.
\item Durmo se, e somente se, trabalho.}
{https://youtu.be/w22pO_WZ_dk}

\quest{Oficial de Justiça 2023 - VUNESP}
{Considere verdadeira a afirmação “Se Marcelo é professor universitário, então Raquel é advogada” e falsa a afirmação “Marcelo é professor universitário e Raquel é advogada”. Nessas condições, é necessariamente verdade que}{
\item Marcelo não é professor universitário.
\item Raquel não é advogada.
\item Marcelo é professor universitário.
\item Marcelo é professor universitário ou Raquel não é advogada.
\item Raquel é advogada.}
{https://youtu.be/kQjrUqqL9eM}

\quest{Técnico Administrativo - IBFC 2023}{ Se todo vegetal é nutritivo e alguns alimentos
são nutritivos, então é correto afirmar que:}{
\item Todo alimento é vegetal
\item Não pode haver alimento que é vegetal
\item Não pode haver alimento que não é vegetal
\item Pode haver vegetal que não é alimento}
{https://youtu.be/ka6Kcb0-Vf8}

\quest{Técnico Administrativo - IBFC 2023}{Se o valor lógico de uma proposição p é verdade e o valor lógico de uma proposição q é falso, então é correto afirmar que:}{
\item A conjunção entre p e q tem valor lógico verdade
\item O bicondicional entre p e q tem valor lógico falso
\item A disjunção entre p e q tem valor lógico falso
\item A disjunção exclusiva entre p e q tem valor lógico falso}
{https://youtu.be/vkQIpeSXZCY}

\quest{Técnico Interno BBTS 2023 - FGV}{Considere a afirmação a seguir.
“Eu fiz dieta e não emagreci.”\\
A negação lógica dessa afirmação é:}{
\item Eu não fiz dieta e não emagreci.
\item Eu não fiz dieta ou emagreci.
\item Eu não fiz dieta e emagreci.
\item Eu não fiz dieta ou não emagreci.
\item Eu fiz dieta e emagreci.}
{https://youtu.be/16Po-EIt1ec}
\end{enumerate}



\end{document}