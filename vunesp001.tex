\section{Oficial de Justiça 2023}


\quest
{No ano de 2022, 3 em cada 8 edifícios comercializados em determinada região foram adquiridos pelo empreen­dimento A&B, que investiu R\$ 1,35 bilhão na compra desses edifícios, ao preço médio de R\$ 15 milhões cada edifício. Dos edifícios não adquiridos pelo empreendimento A&B e que foram comercializados naquela r­egião, o empreendimento R&T adquiriu metade, ao custo total R\$ 1,23 bilhão, o que fez com que o preço médio, por edifício adquirido pela R&T, fosse de}
{\item R\$ 16,3 milhões.
\item R\$ 16,1 milhões.
\item R\$ 16,5 milhões.
\item R\$ 16,4 milhões.
\item R\$ 16,2 milhões.}
{\canal}


\quest
{Considere as informações apresentadas na tabela a seguir, relacionadas à produção de certa peça que é
realizada apenas por máquinas iguais, trabalhando ao mesmo tempo, com a mesma capacidade de produção.\\
\includegraphics[scale=.5]{fig002}\\
Sabendo-se que as informações apresentadas são proporcionais, que em 30/08/2023 o número de máquinas em funcionamento era um quinto maior que o número de máquinas trabalhando no dia seguinte, e que o número de peças produzidas em 31/08/2023 foi quatro terços do número de peças produzidas no dia anterior, é correto afirmar que a carga horária trabalhada no dia 31/08/2023 foi de}
{
\item 8 horas.
\item 7 horas.
\item 9 horas.
\item 8 horas e 30 minutos.
\item 7 horas e 30 minutos.}
{\canal}

\questao{}
{Sabendo-se que é falsidade a afirmação “Se Nora trabalhou, então ela precisa descansar”, assinale a alternativa que apresenta uma afirmação verdadeira.
}
{
\item Nora trabalhou e ela não precisa descansar.
\item Nora não trabalhou e ela não precisa descansar.
\item Nora trabalhou e ela precisa descansar.
\item Nora não trabalhou ou ela precisa descansar.
\item Nora não trabalhou e ela precisa descansar.}
{https://www.youtube.com/watch?v=hcdlzi_qYfc}

\questao{}
{Considere verdadeiras as seguintes premissas:
\begin{enumerate}[I.]
\item Se Carla não é casada ou Pedro não é divorciado, então Cláudio é filho único.
\item Se Sônia é mãe, então Carla não é casada.
\item Se Pedro não é divorciado, então Sergio não é administrador e Gerson é noivo.
\item Cláudio não é filho único.
\end{enumerate}
Uma conclusão que decorre das premissas apresentadas e forma, juntamente com as premissas, um argumento
válido é}
{
\item Gerson é noivo.
\item Sergio não é administrador.
\item Sônia não é mãe.
\item Sônia é mãe.
\item Sergio é administrador.}
{https://www.youtube.com/watch?v=hcdlzi_qYfc}

\questao{}{Sobre um grupo de atletas sabe-se que 15 praticam natação, atletismo e ciclismo, 20 praticam somente n­atação e atletismo, 27 praticam somente natação e c­iclismo, e 25 praticam somente atletismo e ciclismo.\\
Se 70 atletas desse grupo praticam natação, 61 praticam atletismo, e 75 praticam ciclismo, então é verdade que, das alternativas a seguir, a que contém a porcentagem que mais se aproxima da relação entre o número de atletas que praticam um único esporte o número total de atletas desse grupo é
}
{
\item 12\%
\item 18\%
\item 20\%
\item 16\%
\item 14\%}
{https://www.youtube.com/watch?v=hcdlzi_qYfc}

\questao{}
{Na sequência numérica 1, 4, 7, 8, 11,14, 19, 22, 25, 26, 29, 32, 37, ..., o 1º elemento é o número 1. Mantida a regularidade, o 11 111º elemento é o número}
{
\item 33 332.
\item 31 111.
\item 33 115.
\item 33 329.
\item 32 228.}
{https://www.youtube.com/watch?v=hcdlzi_qYfc}

\questao{}
{Considere a seguinte afirmação: “Ou durmo ou trabalho”. Uma negação lógica para a afirmação apresentada é}
{
\item Ou não durmo ou não trabalho.
\item Trabalho ou durmo.
\item Se não durmo, então não trabalho.
\item Não trabalho e não durmo.
\item Durmo se, e somente se, trabalho.}
{https://www.youtube.com/watch?v=hcdlzi_qYfc}

\questao{}
{Considere verdadeira a afirmação “Se Marcelo é professor universitário, então Raquel é advogada” e falsa a afirmação “Marcelo é professor universitário e Raquel é advogada”.\\
Nessas condições, é necessariamente verdade que}
{
\item Marcelo não é professor universitário.
\item Raquel não é advogada.
\item Marcelo é professor universitário.
\item Marcelo é professor universitário ou Raquel não é advogada.
\item Raquel é advogada.}
{https://www.youtube.com/watch?v=hcdlzi_qYfc}




