\chapter{Sequências}

\section{Progressão Aritmética e Progressão Geométrica}
\quest{Técnico Administrativo - IBFC 2023}{Numa progressão geométrica de razão igual a 2, sabe-se que o primeiro termo é igual a 3. Nessas condições, assinale a alternativa que apresenta o quinto termo de uma progressão aritmética cuja razão é igual a 3 e cujo primeiro termo é igual ao segundo termo da progressão geométrica anterior.}{
\item 12
\item 15
\item 14
\item 18}
{https://youtu.be/lNyY7kVNx4U}

\section{Padrões em Sequências}
\quest{Oficial de Justiça 2023 - VUNESP}{ Na sequência numérica 1, 4, 7, 8, 11,14, 19, 22, 25, 26, 29, 32, 37, ..., o 1º elemento é o número 1. Mantida a regularidade, o 11 111º elemento é o número}
{ 
\item 33 332.
\item 31 111.
\item 33 115.
\item 33 329.
\item 32 228.
}{https://youtu.be/whv3ZbhWa9U}

\quest{Técnico Interno BBTS 2023 - FGV}{Dentro de uma caixa são colocadas 6 caixas menores. Depois, dentro de cada uma dessas caixas menores, ou são colocadas 6 caixas menores ainda ou não é colocada caixa alguma. Esse processo se repete um certo número de vezes sendo que a cada vez, dentro das menores caixas, ou são colocadas 6 caixas menores ainda ou não é colocada caixa alguma.
Ao final, sabe-se que há 9 caixas cheias, isto é, caixas onde foram colocadas caixas menores.
O número de caixas vazias é}{
\item 36.
\item 42.
\item 44.
\item 46.
\item 48.}
{https://youtu.be/OkKbixuU2KU}